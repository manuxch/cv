\section{Antecedentes Docentes}

\subsection{Cursos de grado}

\subsubsection{Universidad Tecnológica Nacional}

\textbf{Facultad Regional La Plata}

\years{2022 - } Profesor Titular ordinario - Dedicación Simple. Cálculo Avanzado. Acceso por concurso.

\years{2018 -- 2022} Profesor Titular Interino - Dedicación Simple. Mecánica de Materiales Granulares. Acceso por designación sin concurso.

\textbf{Facultad Regional Buenos Aires}

\years{2014 -- 2022} Profesor Adjunto - Dedicación Simple. Física II. Acceso por concurso.

\years{2005 -- 2014} Profesor Adjunto Interino - Dedicación Simple. Física II. Acceso por designación sin concurso.

\subsubsection{Universidad Nacional de La Plata}
  
\textbf{Facultad de Ciencias Exactas:}
 
\years{2001} Profesor Adjunto Transitorio - Dedicación Simple. Análisis Matemático I. Acceso por concurso.

\years{1999 -- 2002} Jefe de Trabajos Prácticos Transitorio - Dedicación Simple. Análisis Matemático I. Acceso por concurso.

\years{1997 -- 1999} Ayudante Diplomado Transitorio - Dedicación Simple. Análisis Matemático I. Acceso por concurso.

\subsubsection{Universidad Nacional de Rosario}
  
\textbf{Facultad de Ciencias Bioquímicas y Farmacéuticas:} 

\years{1994 -- 1995} Jefe de Trabajos Prácticos Interino - Dedicación Simple. Física II. Acceso por concurso.\\
\years{1995 -- 1996} Ayudante de Primera Interino - Dedicación Simple. Física II. Acceso por concurso. \\
\years{1994 -- 1996} Ayudante de Primera Interino - Dedicación Semiexclusiva. Física I. Acceso por concurso. \\
\years{1993 -- 1994} Ayudante de Primera Interino - Dedicación Simple. Física I. Acceso por concurso. \\
\years{1993 -- 1993} Ayudante de Segunda Interino - Dedicación Simple. Física I. Acceso por concurso. \\
\years{1992 -- 1993} Ayudante de Segunda Interino - Dedicación Simple. Física I. Acceso por concurso. \\

\textbf{Facultad de Ciencias Médicas:}

\years{1994 -- 1995} Jefe de Trabajos Prácticos Interino - Dedicación Simple. Biofísica. Acceso por designación sin concurso. \\

\textbf{Facultad de Ciencias Veterinarias:}

\years{1995 -- 1996} Ayudante de Primera Interino - Dedicación Semiexclusiva. Física Biológica. Acceso por designación sin concurso.\\

\textbf{Facultad de Ciencias Exactas, Ingeniería y Agrimensura:}

\years{1994 -- 1995} Ayudante de Primera Interino - Dedicación Simple. Métodos Numéricos. Acceso por designación sin concurso.

\subsection{Cursos de postgrado dictados}

\years{2008 -- 2024} \textbf{Herramientas computacionales para científicos.}

 Instituto de Física de Líquidos y Sistemas Biológicos. Facultad de Ciencias Exactas y Facultad de Ciencias Astronómicas y Geofísicas, Universidad Nacional de La Plata; Facultad Regional La Plata, Universidad Tecnológica Nacional. Coordinadores: Dr. Manuel Carlevaro, Dr. Luis Pugnaloni (2008 -- 2018), Dr. Ramiro Irastorza (desde 2019). Duración: 70 horas.

\years{2022} \textbf{Physics and Applications of Granular Matter.}

New Jersey Institute of Technology, Department of Mathematical Sciences y Ministerio de Educación de Argentina. Coordinador: Lou Kondic (Distinguished Professor NJIT). Duración: 140 horas.

 \years{2017} \textbf{Herramientas computacionales para la mecánica estadística.}

Universidad Nacional de La Pampa, Facultad de Ciencias Exactas y Naturales. Responsables: Dr. Luis Pugnaloni, Dr. Manuel Carlevaro. Duración: 40 horas.

 
\years{2015} \textbf{Líquidos y Sistemas Desordenados.}

Universidad Nacional de San Luis, Facultad de Ciencias Físico Matemáticas y Naturales. Responsables: Dr. Tomás Grigera, Dr. Manuel Carlevaro. Duración: 20 horas.


 \years{2007, 2010} \textbf{Introducción a Sistemas Dinámicos y Teoría del Caos.}
 
 Escuela de Postgrado y Educación Continua. Facultad de Ingeniería, Universidad Nacional de La Plata. Coordinador: Dr. Fernando Vericat. Duración: 60 horas.
 
 \years{2007} \textbf{Introducción a la programación, al cálculo numérico y a la simulación para científicos.}
 
 Instituto de Física de Líquidos y Sistemas Biológicos. Facultad de Ciencias Exactas, Universidad Nacional de La Plata. Coordinadores: Dr. Luis Pugnaloni, Dr. Manuel Carlevaro. Duración: 70 horas.
