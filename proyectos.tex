\section{Participación en Proyectos de Investigación y Financiamiento}
\subsection{En curso}

\years{2023 -- 2024} \textit{Optimización del consumo de energía en sistemas de aireación de silos}. Proyecto de la Primera Convocatoria del ``Fondo de Innovación Tecnológica de Buenos Aires'', A64 (\$ 7.850.000). \textbf{Director}.

\years{2023 -- 2026} \textit{Experiments and modeling of particle dampers with obstacles}. Otorgado por la Agencia Nacional de Promoción Científica y Tecnológica, PICT-2021-I-A-00294 (\$ 6.480.000). \textbf{Integrante del grupo responsable}. Titular: Luis Pugnaloni.

\years{2022 -- 2024} \textit{Inducción de fibras amiloides por membranas beta-pancreáticas oxidadas en diabetes tipo 2}. Proyecto Plurianual otorgado por CONICET, PIP 11220210100884CO (\$ 1.600.000). \textbf{Director}.


\subsection{Anteriores}

\years{2020 -- 2023} \textit{Flujo y transporte de material granular en sistemas de interés tecnológico}. Otorgado por la Universidad Tecnológica Nacional, MAUTILP0007746TC (\$ 262.000). \textbf{Director}.

\years{2020 -- 2023} \textit{Propiedades estructurales en carga y descarga de silos}. Otorgado por la Universidad Tecnológica Nacional,\\ MAUTNLP0006542 (\$ 154.000). \textbf{Codirector}.

\years{2019 -- 2023} \textit{Estudio de fluidos confinados en sistemas de interés tecnológico}. Proyecto de Investigación de Unidades Ejecutoras - PUE 2018 229 20180100010 CO, otorgado por CONICET (\$ 4.650.000). \textbf{Responsable Científico Técnico}.

\years{2017 - 2022} \textit{Effects of confinement on inhomogeneous systems}. Otorgado por el programa Horizon 2020 Marie Sklodowska-Curie Research and Innovation Staff Exchange de la Comisión Europea, CONIN H2020-MSCA-RISE-2016 Grant Nº 734276 (€ 675.000). \textbf{Investigador}. Responsable: Alina Ciach (Polonia).

\years{2018 -- 2020} \textit{Desarrollo e implementación de una metodología para la evaluación \textit{in vivo} de la calidad ósea}. Otorgado por la Agencia Nacional de Promoción Científica y Tecnológica, PICT 2016-2303, (\$ 409.500). \textbf{Integrante del grupo responsable}. Titular: Ramiro Irastorza.

\years{2018 -- 2020} \textit{Atenuación de vibraciones mediante materiales granulares}. Otorgado por la Agencia Nacional de Promoción Científica y Tecnológica, PICT 2016-2658 (\$ 798.000). \textbf{Integrante del grupo responsable}. Titular: Luis Pugnaloni.

\years{2017 -- 2019} \textit{Estudio de propiedades dinámicas y estructurales de sistemas granulares}. Otorgado por la Universidad Tecnológica Nacional (IFI4434TC) acreditado en el Programa de Incentivos a los Docentes Investigadores (\$ 350.000). \textbf{Director.}

\years{2015 -- 2018} \textit{Transporte y Estabilidad del Agente de Sostén en Fracturas No Convencionales}. Proyecto de desarrollo tecnológico financiado por YPF Tecnología S.A. (\$ 10.032.055), acreditado en el Banco Nacional de Proyectos de Desarrollo Tecnológico y Social del Ministerio de Ciencia, Tecnología e Innovación (PDTS-0256). Titular: Luis Pugnaloni.

\years{2015 -- 2018} \textit{Proyecto de adquisición complementaria ``Plan de Mejoras del Centro de Cálculo del IFLySiB''}. Financiamiento otorgado por el Ministerio de Ciencia, Tecnología e Innovación Productiva (Res. Nro. 054/15) (\$ 168.750). \textbf{Responsable Técnico}.

\years{2014 -- 2016}  \textit{Líquidos clásicos y fermiónicos: Estudio teórico y computacional}. Proyecto plurianual otorgado por el CONICET, PIP 112-201201-00154 (\$ 150.000). Titular: Fernando Vericat.

\years{2013 -- 2016} \textit{Colapso inelástico de medios granulares y descarga de silos}. Otorgado por la Agencia Nacional de Promoción Científica y Tecnológica, PICT 2012-2155 (\$ 1.556.080), \textbf{Codirector}. Titular: Luis Pugnaloni.

\years{2013 -- 2016} \textit{Estudio y análisis de materiales granulares}. Otorgado por la Universidad Tecnológica Nacional (IFI1871) acreditado en el Programa de Incentivos a los Docentes Investigadores, 25/CI01 (\$ 76.000). \textit{Director.}

\years{2013} \textit{Divulgación de actividades científicas a través de CienciaNet} Proyecto de divulgación científico - tecnológica otorgado por el CONICET (\$ 10.000). \textbf{Director}.

\years{2012 -- 2013} \textit{Diseño, implementación y evaluación de situaciones problemáticas abiertas en física básica para ingenieros}. Otorgado por la Universidad Tecnológica Nacional, UTN1535 (\$ 12.500). \textbf{Director}.

\years{2010 -- 2012} \textit{Termodinámica estadística}. Otorgado por la Universidad Nacional de La Plata. Proyecto 11/I153 (\$ 16.000). Titular: Fernando Vericat. 

\years{2009 -- 2011} \textit{Estudio teórico y computacional de líquidos.} Proyecto plurianual otorgado por el CONICET, PIP  112-200801-01192 (\$ 150.000). Titular: Fernando Vericat.

\years{2008 -- 2009} \textit{Propiedades termodinámicas, estructurales y electrónicas de líquidos. Teoría y simulación.} Otorgado por la Agencia Nacional de Promoción Científica y Tecnológica, PICT 2007-00908 (\$ 120.000). Titular: Fernando Vericat.

\years{2006 -- 2009} \textit{Termodinámica Estadística}. Otorgado por la Universidad Nacional de La Plata, Proyecto 11/I108 (\$ 16.000). Titular: Fernando Vericat.

\years{2006 -- 2008} \textit{Teoría y Simulación de Líquidos.} Proyecto plurianual otorgado por el CONICET, PIP Nro. 6240 (\$ 27.000). Titular: Fernando Vericat.

\years{1998 -- 2001} \textit{Termodinámica Estadística.} Otorgado por la Universidad Nacional de La Plata, Proyecto 11/I055. Titular: Fernando Vericat.

\years{1999 -- 2002} \textit{Propiedades Termodinámicas, Estructurales y Electrónicas de Líquidos.} Teoría y Simulación. Otorgado por la Agencia Nacional de Promoción Científica y Tecnológica, PICT 034517. Titular: Fernando Vericat.

\years{1997 -- 1999} \textit{Estudio Mecánico Estadístico de Sistemas Desordenados.} Proyecto plurianual otorgado por el CONICET, PIP Nro. 4690. Titular: Fernando Vericat.
